\documentclass{article}
\title{For Emacs Test}
\author{shi(chunqi.shi@hotmail.com)}
\date{August 2012}
\begin{document}
   
\maketitle
\section{Hello world!}
\emph{Please}:
\begin{itemize}
\item  \textbf{Compile}: Press CTR+C, CTR+C,  Then Return.
\item  \textbf{Read}:    Press CTR+C, CTR+V,  Then Double-Click On the PDF File. 
\end{itemize}

\section{Emacs Commands List} 



C = Control


M = Meta = Alt|Esc



Basics


C-x C-f "find" file i.e. open/create a file in buffer


C-x C-s save the file


C-x C-w write the text to an alternate name


C-x C-v find alternate file


C-x i insert file at cursor position


C-x b create/switch buffers


C-x C-b show buffer list


C-x k kill buffer


C-z suspend emacs 


C-X C-c close down emacs



Basic movement


C-f forward char


C-b backward char


C-p previous line


C-n next line


M-f forward one word


M-b backward one word


C-a beginning of line


C-e end of line


C-v one page up


M-v scroll down one page



Editing


M-n repeat the following command n times


C-u repeat the following command 4 times


C-u n repeat n times


C-d delete a char


M-d delete word


M-Del delete word backwards


C-k kill line



C-Space Set beginning mark (for region marking for example)


C-W "kill" (delete) the marked region region


M-W copy the marked region


C-y "yank" (paste) the copied/killed region/line


M-y yank earlier text (cycle through kill buffer)


C-x C-x exchange cursor and mark



C-t transpose two chars


M-t transpose two words


C-x C-t transpose lines


M-u make letters uppercase in word from cursor position to end


M-c simply make first letter in word uppercase


M-l opposite to M-u



Important


C-g quit the running/entered command


C-x u undo previous action


M-x revert-buffer RETURN (insert like this) undo all changes since last save


M-x recover-file RETURN Recover text from an autosave-file


M-x recover-session RETURN if you edited several files



Online-Help


C-h c which command does this keystroke invoke


C-h k which command does this keystroke invoke and what does it do?


C-h l what were my last 100 typed keys


C-h w what key-combo does this command have?


C-h f what does this function do


C-h v what's this variable and what is it's value


C-h b show all keycommands for this buffer


C-h t start the emacs tutorial


C-h i start the info reader


C-h C-k start up info reader and go to a certain key-combo point


C-h F show the emacs FAQ


C-h p show infos about the Elisp package on this machine



Search/Replace


C-s Search forward


C-r search backward


C-g return to where search started (if you are still in search mode)


M-% query replace


Space or y replace this occurence


Del or n don't replace



s save all listed bookmarks




f show bookmark the cursor is over


m mark bookmarks to be shown in multiple window


v show marked bookmarks (or the one the cursor is over)


t toggle listing of the corresponding paths


w " path to this file


x delete marked bookmarks


Del ?


q quit bookmark list




M-x bookmark-write write all bookmarks in given file


M-x bookmark-load load bookmark from given file



Shell


M-x shell starts shell modus


C-c C-c same as C-c under unix (stop running job)


C-d delete char forward


C-c C-d Send EOF


C-c C-z suspend job (C-z under unix)


M-p show previous commands



DIRectory EDitor (dired)


C-x d start up dired


C (large C) copy 


d mark for erase


D delete right away


e or f open file or directory


g reread directory structure from file


G change group permissions (chgrp)


k delete line from listing on screen (don't actually delete)


m mark with *


n move to next line


o open file in other window and go there


C-o open file in other window but don't change there


P print file


q quit dired


Q do query-replace in marked files


R rename file


u remove mark


v view file content


x delete files marked with D


z compress file

! apply shell command to this file


% d mark files described through regular expression for deletion


% m " (with *)


+ create directory



s toggle between sorting by name or date



Maybe into this category also fits this command:


M-x speedbar starts up a separate window with a directory view



Telnet


M-x telnet starts up telnet-modus


C-d either delete char or send EOF


C-c C-c stop running job (similar to C-c under unix)


C-c C-d send EOF


C-c C-o clear output of last command


C-c C-z suspend execution of command


C-c C-u kill line backwards


M-p recall previous command



Text


Works only in text mode 


M-s center line


M-S center paragraph


M-x center-region name says 



Macro-commands

C-x e execute last definied macro


M-n C-x e execute last defined macro n times


M-x name-last-kbd-macro give name to macro (for saving)


M-x insert-keyboard-macro save named macro into file


M-x load-file load macro


M-x macroname execute macroname



Programming


M C-
 indent region between cursor and mark


M-m move to first (non-space) char in this line


M-; formatize and indent comment


C, C++ and Java Modes


M-a beginning of statement


M-e end of statement


M C-a beginning of function


M C-e end of function


C-c RETURN Set cursor to beginning of function and mark at the end


C-c C-q indent the whole function according to indention style


C-c C-a toggle modus in which after electric signs (like {}:';./*) emacs does the indention


C-c C-d toggle auto hungry mode in which emacs deletes groups of spaces with one del-press


C-c C-u go to beginning of this preprocessor statement


C-c C-c comment out marked area


More general (I guess)


M-x outline-minor-mode collapses function definitions in a file to a mere {...} 


M-x show-subtree If you are in one of the collapsed functions, this un-collapses it 


In order to achive some of the feats coming up now you have to run etags *.c *.h *.cpp (or what ever ending you source files have) in the source directory


M-. (Thats Meta dot) If you are in a function call, this will take you to it's definition 


M-x tags-search ENTER Searches through all you etaged 


M-, (Meta comma) jumps to the next occurence for tags-search 


M-x tags-query-replace yum. This lets you replace some text in all the tagged files 




GDB (Debugger)


M-x gdb starts up gdm in an extra window



Version Control


C-x v d show all registered files in this dir


C-x v = show diff between versions


C-x v u remove all changes since last checkin


C-x v ~ show certain version in different window


C-x v l print log


C-x v i mark file for version control add


C-x v h insert version control header into file


C-x v r check out named snapshot


C-x v s create named snapshot


C-x v a create changelog file in gnu-style



\end{document}